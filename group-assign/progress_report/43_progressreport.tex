\documentclass[onecolumn, draftclsnofoot,10pt, compsoc]{IEEEtran}
\usepackage{graphicx}
\usepackage{url}
\usepackage{setspace}

\usepackage{geometry}
\geometry{textheight=9.5in, textwidth=7in}

% 1. Fill in these details
\def \CapstoneTeamName{Transportation Modeling}
\def \CapstoneTeamNumber{43}
\def \GroupMemberOne{Eytan Brodsky}
\def \GroupMemberTwo{Liang Du}
\def \GroupMemberThree{Samantha Estrada}
\def \GroupMemberFour{Shengjun Gu}
\def \GroupMemberFive{Charles Koll}
\def \CapstoneProjectName{Autonomous vehicle routing in congested transportation network.}
\def \CapstoneSponsorCompany{Oregon State University}
\def \CapstoneSponsorPerson{Haizhong Wang}

% 2. Uncomment the appropriate line below so that the document type works
\def \DocType{Progress Report}

\newcommand{\NameSigPair}[1]{\par
\makebox[2.75in][r]{#1} \hfil 	\makebox[3.25in]{\makebox[2.25in]{\hrulefill} \hfill		\makebox[.75in]{\hrulefill}}
\par\vspace{-12pt} \textit{\tiny\noindent
\makebox[2.75in]{} \hfil		\makebox[3.25in]{\makebox[2.25in][r]{Signature} \hfill	\makebox[.75in][r]{Date}}}}
% 3. If the document is not to be signed, uncomment the RENEWcommand below
%\renewcommand{\NameSigPair}[1]{#1}

%%%%%%%%%%%%%%%%%%%%%%%%%%%%%%%%%%%%%%%
\begin{document}
\begin{titlepage}
    \pagenumbering{gobble}
    \begin{singlespace}
        %\includegraphics[height=4cm]{coe_v_spot1}
        \hfill
        % 4. If you have a logo, use this includegraphics command to put it on the coversheet.
        %\includegraphics[height=4cm]{CompanyLogo}
        \par\vspace{.2in}
        \centering
        \scshape{
            \huge CS Capstone \DocType \par
            {\large\today}\par
            \vspace{.5in}
            \textbf{\Huge\CapstoneProjectName}\par
            \vfill
            {\large Prepared for}\par
            \Huge \CapstoneSponsorCompany\par
            \vspace{5pt}
            {\Large\NameSigPair{\CapstoneSponsorPerson}\par}
            {\large Prepared by }\par
            Group\CapstoneTeamNumber\par
            % 5. comment out the line below this one if you do not wish to name your team
            \CapstoneTeamName\par
            \vspace{5pt}
            {\Large
                \NameSigPair{\GroupMemberOne}\par
                \NameSigPair{\GroupMemberTwo}\par
                \NameSigPair{\GroupMemberThree}\par
                \NameSigPair{\GroupMemberFour}\par
                \NameSigPair{\GroupMemberFive}\par
            }
            \vspace{20pt}
        }
        \begin{abstract}
        % 6. Fill in your abstract
            The purpose of this document is to detail the progress made in the project during Fall term.
            This document discusses progress made each week, along with problems and solutions.
        \end{abstract}
    \end{singlespace}
\end{titlepage}
\newpage
\pagenumbering{arabic}
\tableofcontents
% 7. uncomment this (if applicable). Consider adding a page break.
%\listoffigures
%\listoftables
\clearpage

% 8. now you write!
\section{Recap}
For the “Autonomous vehicle routing in congested transportation network” project, our group purpose is to simulate connected autonomous vehicles(CAV) by building a transportation network model, then introducing both autonomous and human driven models into the system.
Currently, our group has chosen Dijkstra’s Shortest Path algorithm for “path” selection of human driven vehicles and connected autonomous vehicles.
Additionally, each autonomous vehicles will express connectivity by communicating important data with each other, which will influent path selection of CAV.
This algorithm will model itself by using connected autonomous vehicles to collect data (travel time and distance) from the CAV’s around it, then using that data to optimize each connected autonomous vehicles’ algorithm for selecting “path”.
\section{Current Progress}
Currently, all requirements, frameworks, and general structure of the project have been decided.
The next step is to write code that will create a basic working model.
\section{Problems \& Solutions}
\subsection{Week 4}
All our team members met with the client this week.
The client asked us to design GUI simulation.
We decide to focus first on modeling our project so we may have a design in mind for our framework, and worry about the cosmetics of a GUI later.
Client provides us with some connection descriptions for autonomous vehicles and gives some problems, such as how to make an autonomous vehicle recognize a human control vehicle.
This becomes an important idea, as we must learn how to distinguish autonomous vehicles from human driven vehicles.
Currently, we are focusing on reading the literature from clients to decide our direction.
\subsection{Week 5}
We also met with clients this week.
Per client request, each of us introduced a relevant piece of literature we had reviewed the week before.
Pertinent pieces of information were discussed, such as how autonomous vehicles collected information and how autonomous coverage affects traffic.
In the sharing of experience, we know that the intelligence of autonomous vehicles will also affect traffic problems.
In some special cases, 100\% optimal route selection cannot alleviate traffic congestion.
Therefore, based on the articles everyone has read, we find that in the future design, maybe we should gradually adjust the intelligent mode of autonomous vehicles.
In addition, we reviewed the implementation of connected traffic lights into the network, making for a hopeful stretch goal.
\subsection{Week 6}
In this week, we have completed requirement document and team standard.
Everyone has participated in the group meeting.
We discussed some technical details, such as system model, system performance, and user usage.
We all agreed on the key points of the project.
At the same time, we also set up a time schedule about how to develop the project.
\subsection{Week 7}
Each of us needs to write a personal Tech Review.
In the technical discussion of the article, there was a little confusion among the team members on the design direction of the model, so we quickly asked clients.
Then there was no problem in the technical part that everyone was responsible for.
Because we are a research project, each team member has done a literature review.
This allows us to help each other understand the project and speed up the design process.
\subsection{Week 8}
This week, we held a client meeting with clients after a group meeting.
In the group meeting, we discussed what kind of model we had in mind.
With the establishment of the model, some problems arise, such as how to build a city model, how to set the direction and number of lanes, and so on.
We have built a preliminary structure for our model, including basic features for the autonomous model and what functions it will be executing or considering.
In the client meeting, we came to the client with questions about his expectations of the model and what they thought the model would look like, receiving good feedback.
\subsection{Week 9}
This week, we will complete the design document.
In the article, we need to have clear goals and detailed plans about our project.
The article was finished smoothly, and we had no differences of opinion.
We designed a clear framework and a solid design concept.
We also discussed what software to use to compile our code and GUI.
\subsection{Week 10}
We've have completed and submitted the design document.
We have discussed the possibility of implementing MATSim into our design, but upon meeting with the client have settled on our initial decision to use Python for data generation and simulation.
\section{Retrospective}
\begin{tabular}{|p{0.15\linewidth}|p{0.25\linewidth}|p{0.25\linewidth}|p{0.25\linewidth}|} \hline
Week & Positives & Deltas & Actions \\ \hline
4 & Met with the client. Completed the project description. & Client requested a GUI for the simulation. & Clarify with client about GUI specifics and requirements. \\ \hline
5 & Completed the requirements document. Each group member compiled a summary of literature reviews, then presented it to the client. & Must hammer down exact differences expected of an autonomous vehicle, versus a human driven, when implemented into the model. & Clarification asked from client as to what autonomous vehicles should be able to do \\ \hline
6 & Completed the team standards document. Determined who would do which part of the tech review. & Vague idea of what the project will entail, despite having finished the requirements document. & Clarify goals with the client and grad students. \\ \hline
7 & Completed the tech review. & Need to select software to use. Unsure because client’s goals are vague. & Better understanding of technology in our scope \\ \hline
8 & Clarified project goals with the client. Created a basic structure of how the code will be laid out. Resubmitted the requirements document with more accurate and clarifying information. & Client’s requirements are fairly ambitious. Project will require more work than other projects. & Discuss which requirements will be stretch goals. Start looking into software to use. \\ \hline
9 & Determined who would do which part of the design document. & Found software that could potentially do what the group was going to build in code & Discuss options with the client to determine the best course of action. \\ \hline
10 & Completed the design document. Discussed options with the client and decided to remain with old plans because other software is slightly out of scope. With client, decided on MIT license for code. & & \\
\hline
\end{tabular}
\end{document}
