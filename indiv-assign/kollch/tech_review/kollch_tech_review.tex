\documentclass[10pt,letterpaper,draftclsnofoot,onecolumn]{IEEEtran}
\usepackage[margin=0.75in]{geometry}

\begin{document}
\author{Charles Koll - Transportation Modeling (Team 43)}
\title{Tech Review}
\IEEEspecialpapernotice{CS 461, Fall 2018, Role: vehicle connection methods and transportation infrastructure GUIs}
\maketitle

\begin{abstract}
With the inclusion of autonomous vehicles into transportation network models, there is little data on how they will create optimal paths as well as how they will coexist with human driven vehicles. To address this issue, we intend to investigate the pairing of connected autonomous vehicles (CAVs) with a Q-learning algorithm backed by reinforcement learning, aiming to gain data suggesting that vehicle autonomy and the overall infrastructure of transportation may be restructured positively to include multiple intelligent agents. Additionally, this project will explore the impact of CAVs relative to transportation congestion, using a Python based framework and vehicle models to create data on how CAVs behave on a transportation network.
\end{abstract}

\pagebreak

\section{Literature Review: Modeling Traffic Signaling}
An important aspect to transportation modeling is the network infrastructure. That can involve street lengths, number of lanes, traffic direction, intersections, and more. To generally represent those I will discuss GUI frameworks in section 3. There can be additional computation behind the infrastructure as well. For intersections, there are traffic lights that determine the flow of traffic in those nodes of the network. Developing a model without any routing at intersections may be easier to implement, but it could cause significant differences in the model from the real world that could negate any potential information collected from the model, especially when modeling traffic flow and congestion.

There are three primary goals when attempting to model traffic signaling. These are simplicity, efficiency, and realism. With simplicity, there cannot be too much computation time involved in determining when to change signals. If there is too much time taken to determine when to change, it is possible that the optimal time to do so occurs during the computation. Another goal that is easier to compute is efficiency. How quickly can vehicles make it through the intersection? Often this is the main reason for modeling the signals. The final goal is how realistic the modeling is. It is great to have an efficient and simple model, but if it can never be implemented in the real world, then it is useless for modeling.
\subsection{An Agent-Based Learning Towards Decentralized and Coordinated Traffic Signal Control \cite{el2010agent}}
To avoid complexity, one solution is to have each traffic control signal act independently of each other. This prevents complexity often found in multi-variable problems and allows for greater complexity at an individual basis. To optimize efficiency, the authors created a reinforcement learning strategy using Q-learning and genetic algorithms. The reinforcement learning allows for an intersection to adjust its timings based on what it has encountered before to better guess an optimal timing. A common method for implementing this is Q-learning. It allows for various times to have more weight than others, so the signal would be able to adjust its strategy for busier times of the day. A downside to reinforcement learning is that it is only based on what the signal has encountered previously. While this means that the signal will improve its efficiency over time, it also means it will have a harder time dealing with new scenarios. An example of this would be a large sporting event or natural disaster. The signal would not see the type of traffic patterns it can handle, so it would not be able to perform at its most efficient. However, the next time it encounters a similar event it would be better able to gauge how to act. To find more efficient solutions based on a given scenario, genetic algorithms are useful to provide novel possibilities. They introduce various types of randomness that allow for adjustments to the current solution. To verify how realistic this type of modeling is, the authors performed a real-life test on a physical intersection in the city of Toronto. The model performed quite well, with an efficiency much greater than that of a pre-timed signal. While this is promising for the efficacy of the model on a very small scale, it is difficult to determine how a network with many of its intersections fitted with this model would perform. However, considering its efficiency and being relatively realistic, reinforcement learning with Q-learning is a promising possibility for our project.
\subsection{Multi-Agent Reinforcement Learning for Traffic Light Control \cite{wiering2000multi}}
The author used a slightly different form of modeling for their traffic control signals called model-based reinforcement learning. While this is very similar to Q-learning, it differs in a couple of key points. Model-based reinforcement learning requires less computational complexity and can learn more quickly from training data, but it requires discrete input data (time, vehicle locations, etc.) and takes more space. This appears promising from the perspective of efficiency but less so from the perspective of reality. While it seems that this type of training is a computationally easier route for modeling, it is unlikely that it would work in the real world due to it requiring discrete information. Theoretically, inputs that are traditionally continuous could be translated into discrete inputs that function well for the signals. However, that would likely require more computation and negate some of the benefit to its quicker learning. Due to these reasons, I think it unlikely that this type of traffic control signal modeling would work well for our project.

It is interesting to note that both documents used reinforcement learning for the traffic control signals. While the method may be a bit complex and require much training data before it is viable, it is hard to beat an efficiency that improves over time.
\section{Literature Review: Distances Between Connecting Vehicles}
When considering connected autonomous vehicles, one must determine the extent of the connectivity. Internal vehicle models can greatly increase in computational complexity as they take in more data about the vehicles around them. While this complexity can lead to more efficient actions, if it is let to grow too large then it can hinder the responsiveness of the vehicle. To solve this problem most models consider vehicles which can communicate with others in their immediate vicinity (those whose actions are most likely to impact the given vehicle), with a set of protocols in place to handle external factors.
\subsection{Connected Vehicles: Solutions and Challenges \cite{lu2014connected}}
A set of protocols must be able to handle external factors of the environment of transportation infrastructure. Buildings and other objects can block signals between vehicles. Distance can create latency on a connection and weak signals. Differing velocities can change rates at which signals get passed back and forth. All of these are challenges that the protocols must be able to handle. The authors suggest several ways of handling these challenges, which pull from existing technologies in internet networking. These include multiple antennas, frequency division multiplexing, and dynamic spectrum access. The research proposes solutions to environmental challenges, but does not implement these in a model. It is unlikely that any one model could handle the computational complexity to introduce all of these problems, but they are useful notes to consider when determining if a model is realistic enough.
\subsection{Game-Theoretic Control for Robot Teams \cite{emery2005game}}
As explained by the author, computational complexity for games where agents, in our case vehicles, must logically determine the actions of another often becomes intractable for just two agents, much less several. This leads to their suggestion that data passed along the network between connected agents be just enough to optimize to a local maximum rather than global. As noted in the previous research, network connections have their own problems of reliably passing information, so that provides another reason for passing minimal data and/or connecting to fewer vehicles. A local maximum is unlikely to be much worse than a global one, and across many agents it could produce a more efficient solution overall.

For our models we will be sure to consider situations that would be unlikely to work in real-life scenarios and determine suitable workarounds so that they are realistic enough to potentially succeed in real-world testing.
\section{GUI Frameworks for Modeling Transportation Infrastructure}
When developing models to represent vehicles interacting with a transportation infrastructure, it is useful to have a graphical interface to use for presenting results or testing code. The component of the GUI that is used to represent the transportation infrastructure can be built from frameworks.
\subsection{tkinter \cite{tkinter}}
tkinter is a graphical library for Python that wraps Tk. Tk is a bare-bones library that provides fast rendering and full customization for whatever is to be built. Since tkinter is built into Python’s set of standard libraries, it is well-documented and relatively bug-free. Unfortunately, it is probably a poor choice for our project because of how much it can do. It is not meant for network graphical interfaces in particular, so developing a working model for our project would likely take more time than we have. Since the client has made it clear that while a graphical interface would be nice to have, it’s not the primary consideration of the project, there would be too much programming to use tkinter successfully.
\subsection{Fast Network Simulation Setup \cite{fnss}}
Fast Network Simulation Setup is a toolchain that includes a GUI and other tools to rapidly set up a network in Python. It is meant for internet networks, but it could be adapted to work for other types of networks. It includes situations such as congestion and scheduling, which would both be components that we would like to implement in our models. There are two primary downsides to this framework: one, it is meant for internet networks so there may not be exact parallels to what we wish to do, and two, it appears so put together that it would likely be difficult to add functionality such as connectivity of vehicles. While this framework may work for our project, it is unlikely to handle nicely when scaling up.
\subsection{JavaScript \cite{javascript}}
Another option for developing a GUI for our models is to make it a web application. Doing so allows for more independent code development and a larger set of libraries to choose from when building the GUI. A challenge with Python is that it is not generally a language meant for building GUIs. While it is certainly possible, JavaScript is built for that. It appears that this may be the best solution for building the GUI of our simulator. JavaScript does not need to be able to handle the complex calculations of where items exist because it can query that information from the Python backend. The primary downside to using JavaScript is that there would be a network connection between it and the backend, so there could potentially be some latency when transmitting data to be displayed.
\section{Conclusion}
After reviewing various literature and technologies, I have determined what appears to be the best options for our project. With regards to traffic control signals, it’s clear that reinforcement learning is a strong candidate due to its ability to improve over time. For distances between connected vehicles, local optimizations are best, minimizing potential hazards introduced by the environment. Finally, for the framework to a graphical user interface for the transportation infrastructure, it appears that using a language built for developing GUIs such as JavaScript may work best. Our final technologies that we end up using may be different, but this document should provide advantages and disadvantages of several options if they were to be implemented in our project.
\bibliographystyle{IEEEtran}
\bibliography{kollch_tech_rev_bib}
\end{document}
