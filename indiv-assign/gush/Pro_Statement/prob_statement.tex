\documentclass[letterpaper,10pt,draftclsnofoot,onecolumn]{IEEEtran}
\usepackage[margin = 0.75in]{geometry}


\title{Autonomous vehicle routing in congested transportation network.}
\author{Shengjun Gu}
\IEEEspecialpapernotice{CS 461 Fall 2018}



\usepackage{natbib}
\usepackage{graphicx}




\begin{document}

\maketitle

\section{Abstract}

This project is dedicated to applying machine learning to autonomous vehicles in crowded traffic networks and evaluate different route strategies in competitive environments. We will focus on making Autonomous vehicle routing, connected vehicles, machine learning, Q-learning and smart city into a new system. By this system, the Autonomous vehicle will have its own intelligence to analysis which way is the optimal path, to get the passenger to the destination. In addition, the project looked at how machines can achieve the best solution while many intelligent agents try to simultaneously minimize their travel time. The Q-learning algorithm with reverse Q matrix is updated to achieve the optimal path within the grid network, during which different coefficients are evaluated and analyzed based on their impact on algorithm performance.

\pagebreak

\section{Description}
The constant upgrading of information and communication technologies, the Internet of things, and automation has had a major impact on the landscape of transportation. These technologies have raised the prospect of Autonomous Vehicles technology aimed at improving traffic performance in terms of safety, mobility and environmental impact. The concept of driverless vehicles (Autonomous Vehicle) has been around for decades; However, high costs and a lack of proper technology hinder its large-scale production. Even so, the acceleration of research and development over the past decade has made the idea of the Autonomous Vehicle to be fulfilled. However, there is no documented way for Autonomous Vehicle to find and determine its route in the road network. The real-time traffic information system is believed to be the most effective way to convey to drivers and vehicles about travel time and the impending accidents. By communicating such a real-time communication capability between the vehicle and the Autonomous Vehicles and the infrastructure, it may be beneficial for vehicle directional capability and then results in a more efficient and intelligent route-finding algorithm. Therefore, autonomous vehicles in the future, especially the routing behavior, to a certain extent, depends on the centralized control to make full use of the transport network's capacity. Our project is mainly aimed at solving the problem of traffic congestion. We mainly have the following points to do. First of all, we are committed to linking Vehicles and city lights, so that the Vehicle can choose the optimal path, to alleviate traffic congestion problem. Second, we need to help Autonomous Vehicles to design a new algorithm, testing, and optimization algorithm to make artificial intelligence can more quickly calculating the optimal path.


\section{Propose Solution}
The combination of connected and Autonomous Vehicles (CAV) systems with vehicle-to-vehicle (V2V) and vehicle-to-infrastructure (V2I) information exchange could have profound effects on multiple aspects of the current transport landscape, such as requirements and behaviors. The mobility of traffic and environmental concerns have led to considerable attention by the academic and industry sectors. The actual performance at the network level will respond to these new changes and will be significantly affected by specific routes and scheduling algorithms developed for individual autonomous driving vehicles and vehicle fleets. We want to develop a coordinated routing mechanism that allows intelligent vehicles to communicate with each other so that they can make online routing decisions in real time. This strategy may help to avoid possible traffic jams to some extent. Then the process is expanded, and the coordinated online vehicle routing mechanism is developed which also ADAPTS to the information perturbation. We need to find a way for CV vehicles to establish a route coordination group based on their route selection; Each smart vehicle in each group shares traffic information with other members, including its route selection. A control center is established to collect and process real-time information, including general traffic conditions, temporary route selection of vehicles, and travel time prediction of vehicles in interconnected environments. First, we model an AMoD system in the network flow framework, where a customer carrying, and empty rebalanced vehicle is represented as passing through a capacity road network (in this model, congestion effect occurs when the vehicle flows along the road to a critical capacity value). Secondly, based on this theory, we will propose an algorithm with high computational efficiency, which is suitable for time-varying and potentially asymmetrical road networks. Thirdly, through the numerical study of real traffic data, we must verify our hypothesis and prove that the proposed real-time routing and rebalancing algorithm is superior to the most advanced point-to-point rebalancing algorithm in reducing customer waiting time, thus avoiding excessive congestion on the road.


\section{Performance metrics}
We will propose a network flow model for an autonomous on - demand system with capacity. We will develop routing and rebalancing issues, and show that in a symmetrical capacity road network, it is always possible to rebalance vehicles in a coordinated way that does not increase traffic congestion. We might use a model network to show that rebalancing does not increase congestion, even to a modest degree of network asymmetry. We will use this theoretical insight to develop a computationally efficient real-time congestion-aware routing and rebalancing algorithm and verify its performance better than the most advanced point-to-point rebalancing algorithm through simulation. This highlighted the consciousness of traffic congestion in the autonomous control strategy design and implementation of the importance of team. We will create a new routing network to test whether our algorithm is suitable for autonomous and manned vehicles to identify each other. We will then create a smart system to help cars find the best routes to run in the city, thus saving time for vehicles. And we're going to submit the results, give you the theoretical information about how the autopilot system can do the best thing. Finally, we will present the research results in a research report, with our road model, code example, and some theoretical formulas. For the success of the project, we need to submit complete code of various tests, submit test reports, and measure the best configuration of unmanned vehicles, including AI system, algorithm, etc. If we can create an entirely new measurement system, it will be more perfect. If not, we will have to make maximum optimization based on the existing system and propose an optimal environment for Autonomous Vehicles.






\end{document}
