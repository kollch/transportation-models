\documentclass[letterpaper,10pt,draftclsnofoot,onecolumn]{IEEEtran} % changes the margin
\usepackage[margin=0.75in]{geometry}
\usepackage{url}
\begin{document}
\author{Samantha Estrada}
\title{Tech and Literature Review}
\maketitle

\begin{abstract}
    With this review, my team and I aim to acquire insight as to which told will aid us in gaining positive results from the
discussion of how they autonomous vehicles will create optimal paths as well as how they coexist with human driven
vehicles. As a researcher and developer, my role within the team will be to pair connected autonomous vehicles
(CAVs) with a Q-learning algorithm, and we hope to identify vehicle autonomy and their behavior within the overall
infrastructure of transportation, perhaps leading us to a conclusion that their inclusion is beneficial to traffic
congestion. Using a framework and vehicle models to create data on how CAVs behave on a transportation network,
this project will define the problem that autonomous vehicles present in the infrastructure we have already built and
live in.
\end{abstract}
\pagebreak
\begin{section}{Agent Based Modeling}
Firstly, we must identify what we will need to build our means of agent based modeling. We would like something
both easy to use and easy to create a UI with later on if we are able to. This framework sits at the foundation of our
project, as it will be used to test our autonomous vehicle models in a grid network manner.
Our first reaction was to create a Python based framework for our agent based modeling. There currently exists a
framework, “Mesa”\cite{Mesa}
, that is specialized to this exact usage. This tool is also useful as it provides analysis modules to
we can utilize to capture data collection for when our simulation models are run. Additionally, it houses visualization
modules, which we may use to create the UI if we reach that point by pairing it with JavaScript to render the model.
This may be a viable option if we come to a conclusion that we would like to use a framework that is already built for
us, and this framework appears to be one that is versatile and malleable.
Referring to a piece of literature within our team, there is also a developed platform called “MAgent”\cite{MAgent}specifically
created to handle a multitude of intelligent agents, as well as control the amount of intelligent agents active during a
given simulation. This seems useful to our project in the scope of assessing the threshold at which intelligent agents
become beneficial or even detrimental to traffic congestion. This platform uses network sharing, which is important
to us as we would like the vehicles to be connected. MAgent also provides a rendering capability with a sliding scope
for simulations, run on a C++ engine, which appears effective when assigning states and attributes for each agent. The
interface itself is in python and all rewards, actions, and agent symbols can be manipulated using a description
language the team who has built this platform has developed. Based on a Boolean style description of events, this
platform seems useful and straightforward for simple tasks that our vehicle models will be faced with. However, it
does seem that it will need a bit of modification, as it focuses on benefits that come from competitive “prey/predator”
style simulations. However, this model is interesting as it introduces the concept that agents can become more
intelligent working in parallel, which is a concept we have touched on when discussing how connected these connected
vehicles would or could truly be. This platform offers an added insight as to how the agents involved would behave
both individually as societally, which could be interesting to analyze, but not directly necessary at this point in time.
\end{section}

\begin{section}{Traffic Congestion Alleviation}
Secondly, we hope to introduce an effective and integrated method of alleviating traffic congestion. While the purpose
of this project is mainly to simulate and create a behavior of how each autonomous vehicle navigates in a normal
traffic infrastructure, we are also interested in ways to use the vehicles to decrease the amount of traffic congestion.
One technique we may introduce is the implementation of adaptive traffic signal control, introduced in the literature
“An Agent Based Learning Towards Decentralized and Coordinated Signal control”. In this paper, it offers the
inclusion of Q-learning controlled traffic signals, rather than the pre-timed optimized signal control popular today.
This inclusion of reinforcement learning into the traffic signals can offer a level of reaction not commonly used, where
the signals themselves will respond to the fluctuations of traffic flow. This creates a relationship between the signals
and its environment to enable it to act as its own agent, in turn assessing an optimal mapping between states and
actions. This could be beneficial to our project specifically as we hope to create a system that works cognitively with
its environment, using not only individual vehicle agents but the infrastructure as a whole. This literature introduces
some drawbacks as well, such as the fact that implementing a model such as this will create a massive set of possible
number of states for each traffic signal.
Additionally, this paper offers 3 states that a traffic signal through. The first state defines a point in time
where vehicles are approaching the current “green” light direction and the current queue length of those at the “red”
light direction. This state holds a vector of N components, each component representing a “phase”; one vector
component represents the maximum arrivals, while other components are the maximum queue lengths for a given lane. The second state uses queue length to represent delay, creating a vector of N components of maximum queue
lengths. Lastly, the third phase represents the vehicle cumulative delay, which is determined by a vehicle’s total time
spent in a queue. This simplistic model could be useful to integrate as it will offer us another method of data collection
to ensure autonomous vehicles will not experience a heavy delay when interacting with traffic signals. If we do
implement this model, it is possible that it could alleviate traffic congestion by prioritizing reward (low time delay)
whenever and wherever possible.
With the use of these traffic signals, this paper also suggests a decentralized system in which each intersection
represents its own system. Therefore, it displays total reward as taking two points and summing all total cumulative
delays of all vehicles in the current system, the assessing their difference. Vehicles are defined as leaving the system
when they cross the stop line. Therefore, reward may be defined quantitatively as positive or negative. Again, this
type of reward analysis using traffic signals as a data collection device could be something useful in the system we
will be building, as we are interested in the numerical data and the ability to assess autonomous vehicle’s performance
in terms of congestion or traffic delay.
\end{section}

\begin{section}{Vehicle Routing Within Congestion}
Lastly, we intend to cater a routing algorithm to not only find the most optimal route but work beneficially
in congested areas. A literature review on the paper “Routing Autonomous Vehicles in Congested Transportation
Networks: Structural Properties and Coordination Algorithms”
\cite{Routing:2016} offers the concept of re-balancing the vehicles to
ensure given areas will not be overpopulated with intelligent agents. Within this paper, the scenario is presented that
the agents are providing a taxi type service, in which they re-balance themselves into regions of a given area in order
to properly cater to service requests. Our scenario is more individualized, as we seek to model privately owned
autonomous vehicles. Still, this topic of re-balancing raises an important point, as one method of avoiding congestion
is simply to avoid a given area that is becoming overcrowded. The algorithm proposed in the paper utilizes the same
reinforcement learning we've seen and discussed before, using rewards such as fuel usage and maximum maintained
speed to determine the cost of a given route. If the cost of a given route is deemed more beneficial than another, the
vehicle may choose to re-balance itself to that location, even if the distance is greater. Re-balancing also provides the
opportunity for congested areas to become clear, as the volume of traffic does not continue to pile up. This is a viable
point to include when we are considering how to build our algorithm, as we must add this factor of “faster” routes
costing more when they are lacking the right traffic flow.
It is also important to define how our model will recognize when congestion is occurring. This paper also
introduces their own method of allowing their grid network to self-diagnose congestion as soon as it becomes apparent.
Using classical traffic flow theory, this model compares vehicle density on a given road with the free flow phase of
traffic. If the flow of vehicles on the road (determined by the product of vehicle density and speed) reaches a
determined maximum, the vehicles travelling reduce their speeds and the road is flagged for congestion. This is another
important detail we must consider, as we wish for the vehicles to be connected and therefore share data on where and
when congestion occurs so that they may avoid it.
\end{section}	

\bibliographystyle{IEEEtran}
\bibliography{estradas_tech_rev_bib}

\end{document}
