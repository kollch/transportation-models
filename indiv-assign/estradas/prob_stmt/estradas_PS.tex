\documentclass[letterpaper,10pt,draftclsnofoot,onecolumn]{IEEEtran} % changes the margin
\usepackage[margin=0.75in]{geometry}
\begin{document}
\author{Samantha Estrada}
\title{Problem Statement}
\maketitle

 \begin{abstract}
      This document offers a practical solution to the inclusion of autonomous vehicles onto the roadway and how they will attempt to create an optimal route to their destinations. By using a reinforcement algorithm, each autonomous vehicle is able to assess the fastest route based on how outside factors affect how the algorithm performs. This algorithm assesses states, actions and rewards on a network grid, granting favor to the action that provides the greatest reward. This documents also takes into account game theory, which allows for a sense of teamwork between autonomous vehicles, sharing information for optimal navigation and their speeds, so reward can be assessed through one another. By utilizing the Q algorithm and game theory, vehicle autonomy and overall transportation infrastructure may be restructured positively. Additionally, this document defines the problem that autonomous vehicles present in the practical world and discusses how to deal with their navigation among other intelligent vehicles.	
 \end{abstract}

\pagebreak

\section{Problem Definition}
Autonomous vehicles are gaining popularity as they become more realistic, and have granted promise to those who cannot or wish not to drive. While this type of technology is exciting, it also is demanded to be efficient, especially in terms of autonomous public transportation and possible use of these vehicles in delivery services. Above all, these vehicles are demanded to provide a fast and reliable service, succeeding in transporting a customer to their destination in the most optimal way. Relatively, the problem exists with how these vehicles will navigate themselves within areas of congestion; the question arises, will autonomous vehicle presence add to the amount of congestion, or will they be able to adjust positively to it? By ensuring that the vehicles are able to adapt beneficially, not expending unnecessary time and carbon emission, their implementation becomes a contribution to the efficiency of the entire road infrastructure, rather than just a commodity for those that wish for a self driving car. The problem also presents the possibility of conflict within multiple intelligent agents, as they all are working towards the most optimal solution. We must then address the competition between autonomous vehicles themselves, to ensure that they can all achieve an optimal route in their case.

\section{Possible Solutions}
In response to this problem, I propose and inclusion of the “Q algorithm” with game theoretic. This combines the ideology of all autonomous vehicles working together as a team to find an optimal path (game theoretic), and the ability of the “Q algorithm” to instill reinforcement learning onto the vehicles. This algorithm can be simply derived to present states, actions, and rewards. From a given state, a vehicle will assess each action’s reward and the new state associated with taking that reward. The state is then updated based on maximum reward and the variables (state and action) are reset to be assessed again. In the case of this problem, each vehicle will acknowledge other positions on the road and assess the speed of the cars in the given lane or whether there is a turn they need approaching, thus weighing reward to changing or remaining in their state (changing lanes). This algorithm uses a greedy mentality, taking the most optimal solution as soon as it presents itself, in which case, this would enable each automated vehicle to take the lane/route that provides the fastest speed/lowest estimated time arrival to their destination. With the addition of new vehicles, this algorithm would continue in its choice of the greedy algorithm, where one autonomous vehicle takes the optimal solution, then the next vehicle takes the next optimal solution and so on until solutions may be possibly reused. Through game theoretic, each autonomous vehicle can utilize the other to optimize their own route. If a vehicle recognizes that most of the other intelligent agents in its lane are moving to another it may be in its best reward to remain in the lane they are in, as moving lanes may just cause congestion. Through this shared knowledge, it can alert other vehicles of congested areas, so that they may avoid them. By using the Q algorithm to define tables for vehicle adjustments, each autonomous vehicle may be able to conduct its own simulations and therefore create more optimal routes. Ideally, a base framework will be constructed to simulate traffic filled with different vehicle types, later adding built models to run on this framework. By melding an automated model to this simulated framework, we can test the versatility of the automation and the algorithm to ensure that they can actually run with the traffic. By filling the test course with naïve agents and one or a few intelligent ones, we will be able to assess the success rate of autonomous vehicles not only among the naïve agents but among one another as well.

\section{Performance Metrics}
By analyzing the results of these framework integrations, an insight to how current transportation infrastructure could be improved is hoped to be achieved. The roads as they are currently not built to help optimize automated vehicles and assessing their behaviors now could help to create a more sustainable traffic flow. This in turn could affect highways and traffic signals, helping road infrastructure to be better prepared for automated vehicles to navigate. Figures for these paths automated vehicles are taking are also desired, as they could reveal a path that most automated vehicles prioritize, which could give way for advancements to be made on the automated vehicles themselves as well as the roads they are using. As for the automated machines, success will be achieved when their routes prove to be more optimal than regular drivers, and that their decision making skills are actually producing a faster or more efficient route. We must also ensure that the autonomous vehicles can run against themselves, as adding more intelligent agents into the Q algorithm may cause issues. The hope is also to test the efficiency of the Q algorithm and discover its threshold, from which we can produce a statistical analysis of how well the algorithm works and in what environments it works best.

\end{document}
